\chapter{SQue in Kalaman Filter for Measuring SOC and SOH }

Having considered the specifications, the extended-range second-order Incremental A/D converter architecture is proposed to attain the ENOB of 12 or equivalently, an SNR of 72~dB at a sampling clock speed of 80~MHz within the 25 number of clock cycles.

The extended-range approach has been investigated, developed and verified in the Simulink. In order to decide the specifications of the operational amplifiers and the resolution in the quantizer in the \textSigma \textDelta -loop, the sensitivity analysis is carried out where the non-idealities such as low frequency gain and finite GBW were introduced in the architecture and were swept to find out their minimum requirements. 

From the analysis and comparison carried out between the standalone IADC and the extended-range IADC explicitly turns out that, in order to attain a given SNR, the extended-range IADC requires a significantly lower number of clock cycles than a conventional IADC, but the requirements of the op-amps are much more stringent. Moreover, the partitioning of the resolution between IADC and ERADC in extended-range IADCs also affects the op-amp specifications: the higher is the resolution in the ERADC, the higher are the op-amp requirements.

Blocks are then designed in the Cadence environment for the extracted specifications and the Extended-range second-order IADC architecture is developed with ideal blocks and then transistor level blocks. In simulations with architecture with ideal blocks, the SNR obtained from the coarse quantization is around 60~dB and overall it is 86~dB. At transistor level simulations, the coarse quantization offers same SNDR of 60~dB while overall SNDR drops down to 70~dB on account of the harmonics present, however, still maintaining the noise floor at same level. In case of extracted simulations, there is no degradation in the SNDR from the coarse quantization keeping it to 60~dB, nevertheless, overall architecture of ERADC exhibits SNDR of 73~dB with presence of harmonics and SNR of 81~dB. The harmonics present in the output spectrum impels the need of further investigation. 

The characterization of the first chip has been done which involves measurements of the oversampling ADC (only the first-stage of the ERADC without residual ADC) architecture which can operate either as sigma-delta modulator (SD-mode) or incremental ADC (I-mode), in order to allow reading out a single-sensor with maximum performance (SD-mode) or multiple multiplexed sensors with lower performance (I-mode). When configured in SD-mode, the ADC achieves a \textit{SNR} of 73.2~dB with a signal bandwidth of 1.25~MHz and a \textit{SNR} of 66.4~dB with a signal bandwidth of 2.1~MHz, while in I-mode the \textit{SNDR} and \textit{SNR} reduce to 56.4~dB and 59.6~dB, respectively. The \textit{SNR} in SD-mode can be further increased by reducing the bandwidth. The ADC, with a sampling frequency of 80~MHz in both modes, consumes 2.6~mA from a 2.5~V power supply. The peak \textit{FoM} achieved in the two modes is 156.0~dB and 144.7~dB, respectively.


