\section*{Conclusion of Chapter \ref{ch:Current_Measurement}}
The current measurement chapter \ref{ch:Current_Measurement} is sweet and short, but it is essential for this project. The overall active balancing concept depends on the battery soc and the soc is directly dependent on the $i_{batt}$.

The focal point of this chapter is to measure the current flowing in/out to the battery and synchronize measurement acquisition. In reality, this is not a big problem when we are at an integrated circuit solution because we might have a dedicated sensor for the current sense, and we can spare N number of peripherals to read/write data. All current sensors in the IC are synchronized with the master clock. Since this project is implemented at a discrete level it is hard to synchronize multiple discrete current sensors at the same time.
Keeping these scenarios in mind I came up with two different algorithms to synch the current measurements a) Parallel write and sequential read, b) Delayed measurements. The parallel write and sequential read algorithms are very classical, writing all the sensors at the same time by keeping their i2c address the same after they acquired current measurements simultaneously, changing the address of the sensor by their address lines, and reading them sequentially.
This approach could be very simple and elegant, practically this works fine, but I have not employed this procedure in this project. OK!, let me make out some scenarios and see whether this algorithm stands. For instance, while writing sensors simultaneously master does not care whether it has received acknowledgment from all the slaves are not, it simply receives an acknowledgment from any one slave, and it will continue the writing process. If anyone sensor dropped in between that could be a dangerous situation because the master doesn't know what is happening until it will try to read the measurements. By the time master read the current measurements the balancing has been already started, so we could be in trouble if it does not know how much current was flowing. It is a great approach, but it makes less confidence for my application, but till I support this algorithm if there are low-priority measurement applications.
\\\\
The second approach that stepped in is the delayed approach, but this is a completely application-dependent approach. I got the privilege of selecting the INA2XX series to shunt current sensors, they have the feature of programming a delay. I program current sensors with particular delays to align all the measurements and start acquiring data at the same time. The delay is included with reading and writing time as well.
\\\\
I have employed both approaches they work amazingly fine. I do not have a cent percent assurance whether they work universally for applications or not. Yet, they hold strong ground to synchronize the current measurements for BMS applications.
