\chapter*{Introduction}
After more than a century of development, thousands of homes have adopted cars, which have evolved from luxury goods to necessities in modern society since the world's first car was launched. The three pillars of energy efficiency, environmental preservation, and safety are the constant themes in the evolution of automotive technology. Pure electric vehicles, or new energy vehicles, have steadily gained industry attention due to their energy-saving and environmental benefits. Since, the turn of the twenty-first century, when oil prices began to rise and environmental pollution issues, such as "haze," became worse \cite{Real_Time_SOC_Estimation_Based_On_EKF_And_UKF_PTorin_Lei}.

At present, the world heading towards safe and green energy. One of the revolutionary research in energy saving and decreasing global warming is EVs. As the core component of energy-saving vehicles (EVs), the development of battery technology is the key to the industrialization of new energy vehicles. 

The promotion of pure electric vehicles has been hampered by people's worries about the range of electric cars, even though all major automakers are working to promote new energy models. Additionally, the frequent incidents of spontaneous combustion of electric cars in recent years have caused people to pay extra attention to the safety of electric cars. The safe and effective operation of the power battery in electric vehicles relies on a precise assessment of the battery status.

Two categories of battery state can be distinguished: those that can be directly measured, such as voltage, current, temperature, etc.; and those that cannot be directly measured[1] but can only be estimated using specific techniques, such as the battery's state of charge (SOC) and state of health (SOH).